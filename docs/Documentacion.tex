\documentclass[12pt,a4paper]{article}

\usepackage[utf8]{inputenc}
\usepackage{graphicx}
\usepackage[T1]{fontenc}
\usepackage[spanish]{babel}
\usepackage[hidelinks]{hyperref}
\usepackage{listings}
\usepackage{xcolor}
\usepackage{titlesec}
\usepackage[a4paper, margin=2cm]{geometry}

\lstset{
    language=C,
    basicstyle=\ttfamily\small,
    keywordstyle=\color{blue}\bfseries,
    commentstyle=\color{green!50!black},
    stringstyle=\color{red},
    numbers=left,
    numberstyle=\tiny,
    stepnumber=1,
    numbersep=5pt,
    breaklines=true,
    frame=single,
    tabsize=4
}

\begin{document}

\begin{titlepage}
    \centering
    \vspace*{1cm}

    \includegraphics[width=0.20\textwidth]{escudounrc.jpg} \\[1cm]

    \textsc{\Large Universidad Nacional de Río Cuarto}\\[0.2cm]
    \textsc{\large Facultad de Ciencias Exactas, Físico - Químicas y Naturales}\\[0.2cm]
    \textsc{\normalsize Departamento de Computación}\\[4cm]

    {\huge \bfseries TDS25 - Documentación del Proyecto}\\[8cm]
    \textbf{Autores:} Jeremias Avaro, Mateo Cornejo, Maximo Marquez Regis \\[0.3cm]
    \textbf{Materia:} Taller de Diseño de Software \\[0.3cm]
    \textbf{Docente:} Francisco Bavera \\[0.3cm]
    \textbf{Año:} 2025
    \date{}

    \vfill
\end{titlepage}

\tableofcontents
\newpage

\section{Introducción}
Este documento recopila las distintas etapas de desarrollo del compilador
realizado como proyecto académico. Se detallan las tareas llevadas a cabo
por cada integrante, las decisiones de diseño adoptadas (y aclaraciones), decisiones clave y y los problemas detectados.

El objetivo es documentar de forma progresiva la evolución del compilador,
dejando registro tanto de los aspectos técnicos como de la organización
del trabajo en equipo.


\section{Primera etapa}

\subsection*{División de la primera etapa}
\begin{itemize}
    \item \textbf{Jeremias Avaro:} inicialización del proyecto y primeros avances del lexer y la gramática.
    \item \textbf{Mateo Cornejo:} refinamiento de lexer y gramática. Creación de tests.
    \item \textbf{Máximo Marquez Regis:} completó la gramática casi en su totalidad. Integración del parser y lexer.
    \item Todos: discusión de decisiones de diseño, revisión de código y documentación.
\end{itemize}

\subsection*{Decisiones de diseño y aclaraciones}
\begin{itemize}
    \item Los comentarios se ignoran en el lexer para simplificar la gramática del parser.
    \item Se optó por separar el manejo de errores en un módulo propio (\texttt{error\_handling.c/h}) para centralizar la gestión de mensajes.
    \item Se decidió que el lexer cuente tokens y líneas para facilitar el reporte de errores y estadísticas.
\end{itemize}

\subsection*{Diseño y decisiones clave}
\begin{itemize}
    \item La gramática permite declaraciones de variables, métodos y bloques de código.
    \item Se definió un \textbf{union} para manejar tokens con valores enteros o cadenas.
    \item Las expresiones incluyen operadores aritméticos, relacionales y lógicos con precedencias adecuadas.
    \item Se implementó un sistema de reporte de errores léxicos y sintácticos.
    \item El parser está preparado para ser extendido con análisis semántico y generación de código.
    \item Se utilizó un Makefile para automatizar la compilación y facilitar la integración de nuevos módulos.
    \item Se crearon tests básicos para validar el funcionamiento del lexer y parser.
\end{itemize}

\subsection*{Problemas conocidos}
\begin{itemize}
    \item Algunos casos límite no han sido exhaustivamente probados.
    \item El manejo de comentarios multilinea podría fallar en comentarios no cerrados.
\end{itemize}

\newpage

\section{Segunda etapa}

\subsection*{División de la segunda etapa}
\begin{itemize}
    \item \textbf{Mateo Cornejo:} implementación del AST incluyendo la definición de estructuras y métodos asociados. Desarrollo de funciones para la visualización del AST. Refactorización del manejo de errores hacia el módulo \texttt{error\_handling}.
    \item \textbf{Jeremias Avaro:} inicialización de la tabla de símbolos, dejando planteada la estructura y la idea principal de su funcionamiento.
    \item \textbf{Máximo Marquez Regis:} finalización de la implementación de la tabla de símbolos, realizando ajustes y refinamientos en funcionalidades complementarias.
    \item Todos: participación en la discusión de decisiones de diseño, testeo y documentación.
\end{itemize}

\subsection*{Decisiones de diseño y aclaraciones}
\begin{itemize}
    \item Se definió la función \texttt{my\_strdup} para duplicar strings, evitando el uso de \texttt{strdup}.
    \item Se creó un enumerado para los tipos de nodos del AST (\texttt{AST\_NODE\_TYPE}). 
    \item En la estructura \texttt{AST\_NODE}, se empleó una \texttt{union} para almacenar información específica según el tipo de nodo.
    \item Se diseñó la estructura \texttt{AST\_NODE\_LIST} para manejar listas de sentencias, declaraciones y argumentos.
    \item Se creó la carpeta \texttt{utils} para funciones de utilidad como \texttt{my\_strdup}.
\end{itemize}

\subsection*{Diseño y decisiones clave}
\begin{itemize}
    \item Se creó el enumerado \texttt{AST\_TYPE} para representar los distintos tipos de nodos.
    \item Se utilizó una \texttt{union} en \texttt{AST\_NODE} para reducir redundancia en la representación de nodos.
    \item Se implementó \texttt{free\_mem} para liberar memoria del AST de manera recursiva.
    \item Se integraron los constructores de nodos en las acciones semánticas del parser.
    \item Se cambió la gramática a recursión por la izquierda para construir listas de statements y declaraciones. Esto permite almacenarlos e imprimirlos directamente en el orden fuente.
    \item Se organizó el proyecto con la carpeta \texttt{utils}.
    \item Se integró la impresión del AST y de la tabla de símbolos al final del análisis sintáctico.
\end{itemize}

\subsection*{Problemas conocidos}
\begin{itemize}
    \item Falta testear exhaustivamente la correcta construcción y recorrido del AST con múltiples tipos de nodos.
\end{itemize}


\end{document}
